\documentclass[a4paper, 12pt]{article}
\usepackage[T1]{fontenc}
\usepackage{polski}
\usepackage[utf8]{inputenc}
\usepackage[polish]{babel}
\usepackage[margin=1in]{geometry}
\usepackage{graphicx}
\usepackage{wrapfig}
\usepackage{fancyhdr}
\usepackage{lastpage}
\usepackage[final]{pdfpages}
\usepackage{indentfirst}
\usepackage[ddmmyyyy]{datetime}
\usepackage{hyperref}
\usepackage{listings}

\renewcommand{\dateseparator}{/}
\fancyhead{}
\renewcommand{\headrulewidth}{0pt}

\pagestyle{fancy}
\cfoot{\thepage\hspace{1pt}/\hspace{2pt}\pageref{LastPage}}


\begin{document}


\begin{wrapfigure}{L}{20px}
%\includegraphics[width=1.5cm,height=1.3cm,keepaspectratio]{logo_ee.png}
\includegraphics[height=1.5cm,keepaspectratio]{WE-znak.png}
\end{wrapfigure}

%Politechnika Warszawska 
\hfill Data utworzenia: \today

%Wydział Elektryczny
\hfill Ostatnia modyfikacja: \today

\hfill Wersja: AZ %AZ to wersja skończona, potem lecimy B1,B2...BZ


\quad
\begin{center}
\center \Huge Tytuł
\center \large Podtytuł
\vspace{0.5cm}\\
\small Autorzy: B W, \underline{Jereczek Michał}, Ł J, M P, W K
\vspace{0.1cm}\\
\small Weryfikacja: Ł J
\end{center}

\tableofcontents
\pagebreak

\section{Rozdział}
Spójrz na przykładowy kod w Listingu \ref{lst:kod}. Fajny prawda?

\begin{lstlisting}[language=Python,frame=single,caption=Skrypt obliczający obciążenie cieplne oraz moment znamionowy,label={lst:kod},   showstringspaces=false]
import math

tx = [5, 4, 6, 3, 3, 4, 15, 20, 5, 5, 4, 26, 10, 4, 4, 5, 31
, 6, 3, 3, 4, 18, 4, 7, 36, 10, 4, 11, 3, 4, 28, 5]
Ix = [0, 140, 0, 115, 175, 290, 120, 0, 170, 0, 160, 150, 0
, 180, 230, 315, 130, 0, 160, 205, 270, 0, 160, 210, 190, 0
, 130, 0, 140, 200, 130, 0]


\end{lstlisting}


\pagebreak
\section{Historia zmian}
\begin{center}
    \begin{tabular}{ | p{0.5cm} | p{4.5cm} | p{6cm} | p{2cm} |p{1.2cm} |}
    \hline
    Nr. & Osoba & Zmiana & Data & Wersja 
    \\ \hline
    3. & Michał Jereczek & Aktualizacja loga itp & 12/11/2017  & A3
    \\ \hline
    2. & Michał Jereczek & Dodanie daty do historii zmian & 30/10/2015  & A2
    \\ \hline
    1. & Michał Jereczek & Dodanie szablonu dokumentu formalnego & 22/10/2015  & A1
    \\ \hline
    \end{tabular}
\end{center}


\end{document}